% Copyright (C) 2022 MeteoSwiss,
% Originally written by F.P.A. Vogt; frederic.vogt@meteoswiss.ch
%
% This file may be distributed and/or modified under the conditions
% of the BSD-3-Clause License.
% The terms of this license are available at:
%
% https://opensource.org/licenses/BSD-3-Clause
% 
% SPDX-License-Identifier: BSD-3-Clause
%
% This file contains the reference TikZ commands to draw the metsymb symbol: STRATUS 
%
% Reference: https://cloudatlas.wmo.int/en/abbr-and-symbols-of-clouds-table-genera-species.html
%
% Created 28.08.2022; frederic.vogt@meteoswiss.ch

% Import the least amount of black magic possible
\documentclass[tikz, border=0cm, crop=true]{standalone}
\usetikzlibrary{calc}

% Load the default metsymb TikZ styles
% Copyright (C) 2022 MeteoSwiss,
% Originally written by F.P.A. Vogt; frederic.vogt@meteoswiss.ch
%
% This file may be distributed and/or modified under the conditions
% of the BSD-3-Clause License.
% The terms of this license are available at:
%
% https://opensource.org/licenses/BSD-3-Clause
% 
% SPDX-License-Identifier: BSD-3-Clause
%
% This file contains the reference TikZ styles, that ought to be used by all metsymb symbols to ensure a uniform look.
% 
% In other words, TikZ styles should *never* be defined in individual tikzpictures, but here instead, where they can be shared and access by all.
% Doing so should ease the global modification of styles for all symbols at once.
%

\tikzstyle{every picture}=[line width=1.4pt, color=black]

% Should a specific subset of symbol ever require some custom style adjustment (which is likely), they can be defined here as follows:
% \tikzstyle{fancy symbol style}=[line width=2.0pt, color=black, ...]


% Start the serious stuff
\begin{document}
\begin{tikzpicture}

% Clip to the 0-1 area, to ensure correct scaling occurs during the import in FontForge.
\clip (0,0) rectangle (1,1);

% Uncomment the line below to show the margins that should be kept clear around the symbol
%\draw [red] (0.025,0.025) rectangle (0.975, 1.25);

% ===== KEEP THE GLYPH COMMANDS BELOW THIS LINE =====
%
% --- Guidelines ---
%
% Stick to the following rules to attain a metsymb-compatible look:
% 0) The default line width has a thickness of 0.05
% 1) The maximum x extent should be: 0.05 - 0.95
% 2) The maximum y extent should be: 0.1 - 1
% 3) The glyphs should be centered (vertically) around: y = 0.55

\draw  (0.05, 0.55) -- (0.41, 0.55);
\draw  (0.59, 0.55) -- (0.95, 0.55);

% ===== KEEP THE GLYPH COMMANDS ABOVE THIS LINE =====

% Finish up
\end{tikzpicture}
\end{document}